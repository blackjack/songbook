\section{Ляпис Трубецкой - Грай}
\begin{guitar}
[Em]Нед ля ракі, дзе не мае броду
[Hm]Шэрыя быкі танчуть карагоды.
[C]Ланцугі у начы адлівають златам.
[Am]Воран там крычыть, звоукам родным братам.
Палыхае там вогнішча да неба.
П’ють яны віно, заядають хлебам.
Песні ім пають старыя цыганкі.
Б’ють капытам, б’ють.
А у небе маланкі.

Припев:

[G]Гэтыя быкі мають сваю прауду:
[D]Ім ня трэба сонца, темры ім багата.
[Em]Ім вясны ня трэба, ім зімы паболей,
[C]Каб ты хлопец спау на пячы у няволе.
Грай, шукай у снах юнацтва свае мары.
Грай, гукай вясны зяленай теплай чары.
Грай, спявай дружна песні райскай волі.
Грай, грай, гані быкоу – вярнецца доля.

[Em]Напілісь быкі, скачуть па краіне,
[Hm]Топчуть ручнікі капытамі у гліне.
[C]Скачуть па дварах, адчыняють хаты,
[Am]Хто не пахавауся — буде вінаваты.

Припев:

[G]Гэтыя быкі мають сваю прауду:
[D]Ім ня трэба сонца, темры ім багата.
[Em]Ім вясны ня трэба, ім зімы паболей,
[C]Каб ты хлопец спау на пячы у няволе.

\end{guitar}