\section{Вій - Колискова}
\begin{guitar}
КОЛИСКОВА

Люлі-люлі, спатоньки, мої янголятонька,
Місяць вже на чатах, скоро темна ніченька на землю ляже…
Стою над колискою: що ж вам дати їстоньки?
Третій день у хаті висить морок голоду, а мати каже:

Нащо заколисуєш? Віднеси ти їх у ліс – 
Хай там ніч їх гойда на чорнім крилі!
Не хрестом – так вилами Бог тебе помилує:
Як ми – прах, хай земля іде до землі.

Щоб блукали росами ніженьками босими,
Від весни до осені мигтіли у траві зеленим сяйвом,
А як стане холодно – сніг розстеле полотно…
Вам не жити все одно – помрете з голоду, і мати каже:

Нащо заколисуєш? Віднеси ти їх у ліс – 
Хай там ніч їх гойда на чорнім крилі!
Та закидай листячком – в мене сил не вистачить:
Всі ми – прах, хай земля іде до землі.

Пташенята любії ще раз приголубити…
Плакати від голоду ви вже не будете – лиш я заплачу:
Ви, спасибі татові, стали потерчатами,
Будете кричати ви під вікнами вночі, а я вам скажу:
Люлі-люлі, спатоньки,
Люлі-люлі, спатоньки,
Люлі-люлі, спатоньки…
\end{guitar}