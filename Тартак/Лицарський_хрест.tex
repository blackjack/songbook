\section{Тартак - Лицарський хрест}
\begin{guitar}
Вступ: Gm E\guitarFlat\ Gm E\guitarFlat\ Gm E\guitarFlat\ D E\guitarFlat

Дихає л[Gm]іс,
Пташка на г[Cm]іллі
Пісню спів[B]ає, що тішить мій сл[D/F#]ух.
Я довго ріс -
Йшов через цілі,
Що тіло гартують і зміцнюють дух.
Тиха роса
Зіб'ється з трав
Криком "вперед!", дружним тупотом ніг.
Я тут знайшов
Те, що шукав,
Славу здобув і себе переміг!

Приспів:

Мій лицарський хр[Gm]ест –
Моя нагор[Eb]ода
За те, що не вп[Cm]ав, за те, що не вт[D/F#]ік!
Мій лицарський хр[Gm]ест –
Яскрава приг[Eb]ода,
Що буде трив[Cm]ати в мені цілий в[D/F#]ік!
Мій лицарський хр[Gm]ест...

Вступ.

Плинуть роки,
Їх заметілі
Скроні мої пофарбують у сніг.
Я, завдяки
Шрамам на тілі,
В пам'ять свою закарбую усіх
Друзів моїх
Та ворогів -
Кого любив і кого вбивав...
Може чогось
Я не зумів,
Та не згубив, не програв, не продав...

Приспів.

Тим, що загинули, й тим, що вижили,
Слово своє вдячно присвячую.
Хай ворог зиркає очима хижими,
Нехай гарчить – мені не лячно!
Хто вріс корінням, той не зламається.
Хто має стержень, той не зігнеться.
Любов до матері – найкраща порадниця.
Любов до вітчизни – ідея серця.
Лицарський хрест – відзнака для обраних –
Не завжди на грудях, а в діях і звершеннях.
Для тих, що в боях ставали хоробрими.
Для тих, що в атаки здіймалися першими.
Не відступитися від слова сказаного –
Дерти руками, зубами гризти.
У світі багато брудного й заразного,
Але той, хто хоче, залишається чистим!
\end{guitar}
